\documentclass{61210}

\author{Artem Laptiev}
\problemset{1}


\begin{document}



\problem{1}{Ari Stotle, and \url{en.wikipedia.org/wiki/Mathematics}}

\section{Formatting examples for figures, pseudocode, etc}

\begin{align*}
  x + y & = z 
\end{align*}

\begin{enumerate}[a.]
  \item Item 1
  \item Item 2
\end{enumerate}

\subsection{Tables}

Many other table packages and options exist but here is one example:\\\\

\begin{tabularx}{0.8\textwidth} {
  | >{\raggedright\arraybackslash}X
  | >{\centering\arraybackslash}X
  | >{\raggedleft\arraybackslash}X | }
 \hline
 item 11 & item 12 & item 13 \\
 \hline
 item 21  & item 22  & item 23  \\
\hline
\end{tabularx}

\subsection{Images}

\begin{figure}[h]
\begin{center}
\includegraphics[width=0.65\textwidth]{images/placeholder.png} % Include the image placeholder.png
\caption{Figure caption.}
\end{center}
\end{figure}

\subsection{Code Blocks and Algorithm Pseudocode}

% Including Python code blocks:

\begin{lstlisting}
json
{
  "6.141": "normal",
  "16.405": "woke",
  "no_sleep": "spoke"
}

def do_something_productive():
  if not_productive:
    do_work()
  else:
    cry()
\end{lstlisting}

% Using the algorithm2e package for pseudocode:

\begin{algorithm}[H]
\SetAlgoLined
 \While{alive}{
  \eIf{sleepy}{
   sleep\;
   }{
   eat\;
  }
 }
 \caption{caption}
\end{algorithm}

% Using the algorithmic package for pseudocode:

\begin{algorithmic}
\STATE $i\gets 10$
\IF {$i\geq 5$}
        \STATE $i\gets i-1$
\ELSE
        \IF {$i\leq 3$}
                \STATE $i\gets i+2$
        \ENDIF
\ENDIF
\end{algorithmic}



\end{document}

